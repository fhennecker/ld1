\documentclass[a4paper]{article}

\usepackage[utf8]{inputenc}
\usepackage[T1]{fontenc}
\usepackage[french]{babel}
\usepackage{fullpage}
\usepackage{hyperref}
\usepackage{amsmath}
\usepackage{amssymb}
\usepackage{upgreek}
\usepackage{color}
\usepackage[]{algorithm2e}
\usepackage{stmaryrd}
\usepackage{graphicx}
\usepackage{float}
\usepackage{multirow}

\title{
    Assignment 1 : Game Theory\\
    \small INFO-F-409 - Learning Dynamics
}
\author{Florentin \bsc{Hennecker} (ULB 000382078)}
\date{}


\begin{document}
\maketitle

\section{The Hawk-Dove game}
Let's assume that the row player escalates with probability $p$ and that it 
displays with probability $1-p$. Similarly, the column player escalates with 
probability $q$ and it displays with probability $1-q$. We can view the game
in a more standard way :

\begin{table}[H]
\begin{tabular}{c|c|c|c}
	& & \multicolumn{2}{c}{\textbf{Column player}} \\
	\hline
	& & Escalate $q$ & Display $1-q$ \\
	\textbf{Row player} & Escalate $p$ & 
		$\frac{V-D}{2}$\textbackslash$\frac{V-D}{2}$ & 
		V \textbackslash $0$\\
	\cline{2-3} & Display $1-p$ & $0$ \textbackslash $V$ &
		$\frac{V}{2}-T$ \textbackslash $\frac{V}{2}-T$\\
\end{tabular}
\end{table}

\paragraph{1} For the row player, when are the expected payoffs for each 
strategy equal?
$$ q\frac{V-D}{2} + (1-q)*V = q*0 + (1-q)*(\frac{V}{2}-T) $$
$$ q\frac{V-D}{2}+V-qV = \frac{V}{2}-T -q\frac{V}{2} + qT $$
$$ (\frac{V-D}{2} - V + \frac{V}{2} - T)q + V - \frac{V}{2} + T = 0 $$
$$ (-D-2T)q + V+2T = 0 $$
$$ q = \frac{-V-2T}{-D-2T} $$

The development is the same for the column player so :
$$ p = \frac{-V-2T}{-D-2T} $$

We can assume that $V$, $D$ and $T$ are positive. This means that \textbf{for
any given $T$} :
\begin{itemize}
	\item if $V = D$ : $p = q = 1$, so there is a pure 
		strategy nash equilibrium in (Escalate, Escalate).
	\item if $V > D$ : $ p = q > 1 $, so there is a pure
		strategy nash equilibrium in (Escalate,	Escalate).
	\item if $V < D$ : $ 0 < p = q < 1 $, there is a mixed strategy Nash
		equilibrium in \\
		$\{( \frac{-V-2T}{-D-2T}, 1-\frac{-V-2T}{-D-2T}), 
		( \frac{-V-2T}{-D-2T}, 1-\frac{-V-2T}{-D-2T})\}$. 
\end{itemize}

\paragraph{2} Displaying becomes more beneficial than escalating for the
row player when : 
$$ 0*q + (\frac{V}{2} - T) (1-q) > q\frac{V-D}{2} + (1-q)*V$$
so, if we refer to the previous development : 
$$ q <  \frac{-V-2T}{-D-2T} $$ 
which means that it is more beneficial for the row (resp. column) player to 
display when the column (resp. row) player escalates with a probability $q$
(resp. $p$) lower than $\frac{-V-2T}{-D-2T}$. 

% \begin{tabular}{c|c|c}
% 	& \multicolumn{2}{c}Display} \\
 %    \hline
  %%    Escalate & 1 & 2 \\
   %  \cline{2-3} & 3 & 4 \\
% \end{tabular}

\end{document}
