\documentclass[a4paper]{article}

\usepackage[utf8]{inputenc}
\usepackage[T1]{fontenc}
\usepackage[french]{babel}
\usepackage{fullpage}
\usepackage{hyperref}
\usepackage{amsmath}
\usepackage{amssymb}
\usepackage{upgreek}
\usepackage{color}
\usepackage[]{algorithm2e}
\usepackage{stmaryrd}
\usepackage{graphicx}
\usepackage{float}
\usepackage{multirow}
\usepackage[table]{xcolor}
\title{
    Assignment 1 : Game Theory\\
    \small INFO-F-409 - Learning Dynamics
}
\author{Florentin \bsc{Hennecker} (ULB 000382078)}
\date{}


\begin{document}
\maketitle

\section{The Hawk-Dove game}
Let's assume that the row player escalates with probability $p$ and that it 
displays with probability $1-p$. Similarly, the column player escalates with 
probability $q$ and it displays with probability $1-q$. We can view the game
in a more standard way :

% \begin{table}[H]
% \begin{tabular}{c|c|c|c}
	% & & \multicolumn{2}{c}{\textbf{Column player}} \\
	% % \hline
	% & & Escalate $q$ & Display $1-q$ \\
	% \textbf{Row player} & Escalate $p$ & 
		% $\frac{V-D}{2}$\textbackslash$\frac{V-D}{2}$ & 
		% V \textbackslash $0$\\
	% \cline{2-3} & Display $1-p$ & $0$ \textbackslash $V$ &
		% $\frac{V}{2}-T$ \textbackslash $\frac{V}{2}-T$\\
% \end{tabular}
% \end{table}

\paragraph{1} For the row player, when are the expected payoffs for each 
strategy equal?
$$ q\frac{V-D}{2} + (1-q)*V = q*0 + (1-q)*(\frac{V}{2}-T) $$
$$ q\frac{V-D}{2}+V-qV = \frac{V}{2}-T -q\frac{V}{2} + qT $$
$$ (\frac{V-D}{2} - V + \frac{V}{2} - T)q + V - \frac{V}{2} + T = 0 $$
$$ (-D-2T)q + V+2T = 0 $$
$$ q = \frac{-V-2T}{-D-2T} $$

The development is the same for the column player so :
$$ p = \frac{-V-2T}{-D-2T} $$

We can assume that $V$, $D$ and $T$ are positive. This means that \textbf{for
any given $T$} :
\begin{itemize}
	\item if $V = D$ : $p = q = 1$, so there is a pure 
		strategy nash equilibrium in (Escalate, Escalate).
	\item if $V > D$ : $ p = q > 1 $, so there is a pure
		strategy nash equilibrium in (Escalate,	Escalate).
	\item if $V < D$ : $ 0 < p = q < 1 $, there is a mixed strategy Nash
		equilibrium in \\
		$\{( \frac{-V-2T}{-D-2T}, 1-\frac{-V-2T}{-D-2T}), 
		( \frac{-V-2T}{-D-2T}, 1-\frac{-V-2T}{-D-2T})\}$. 
\end{itemize}

\paragraph{2} Displaying becomes more beneficial than escalating for the
row player when : 
$$ 0*q + (\frac{V}{2} - T) (1-q) > q\frac{V-D}{2} + (1-q)*V$$
so, if we refer to the previous development : 
$$ q >  \frac{V+2T}{D+2T} $$ 
which means that it is more beneficial for the row (resp. column) player to 
display when the column (resp. row) player escalates with a probability $q$
(resp. $p$) higher than $\frac{V+2T}{D+2T}$. 

% \begin{tabular}{c|c|c}
% 	& \multicolumn{2}{c}Display} \\
 %    \hline
  %%    Escalate & 1 & 2 \\
   %  \cline{2-3} & 3 & 4 \\
% \end{tabular}

\section{Which social dilemma?}
Let us first find player 1's best responses (highlighted in red) :\\ 
\begin{table}[H]
\begin{tabular}{c|c|c|c|c|c|c|c|c}
	& (C,C,C) & (C,C,D) & (C,D,C) & (C,D,D)
	& (D,C,C) & (D,C,D) & (D,D,C) & (D,D,D)\\
	\hline
	C & $\frac{1}{3} * (2 + 5 + 2 ) = 3 $ &
		\cellcolor{red!25} $\frac{8}{3}$ &$\frac{4}{3}$ & 1 &
		$\frac{7}{3}$ & 2 & $\frac{2}{3}$ &
		$\frac{1}{3}$ \\
	\hline
	D & \cellcolor{red!25} 4 & $\frac{7}{3}$ & 
		\cellcolor{red!25} $\frac{11}{3}$ & \cellcolor{red!25} 2
		& \cellcolor{red!25} $\frac{8}{3}$ & \cellcolor{red!25} 3 
		& \cellcolor{red!25} $\frac{7}{3}$ & 
		$\cellcolor{red!25} \frac{2}{3}$ \\
\end{tabular}
\end{table}

These are player 2's best responses :\\
\begin{table}[H]
\begin{tabular}{c|c|c|}
	& \textbf{C} & \textbf{D}\\
	\hline
	\textbf{C} & 2,2 & \cellcolor{red!25} 0,5 \\
	\hline
	\textbf{D} & 5,0 & \cellcolor{red!25} 1,1 \\
\end{tabular}
\hfill
\begin{tabular}{c|c|c|}
	& \textbf{C} & \textbf{D}\\
	\hline
	\textbf{C} & \cellcolor{red!25} 5,5 & 0,2 \\
	\hline
	\textbf{D} & 2,0 & \cellcolor{red!25} 1,1 \\
\end{tabular}
\hfill
\begin{tabular}{c|c|c|}
	& \textbf{C} & \textbf{D}\\
	\hline
	\textbf{C} & 2,2 & \cellcolor{red!25} 1,5 \\
	\hline
	\textbf{D} & \cellcolor{red!25} 5,1 & 0,0 \\
\end{tabular}
\end{table}

From these best responses we can deduct the Bayesian Nash equilibrium
(D,(D,D,C))

\section{Sequential truel}
\begin{figure}[H]
	\includegraphics[width=\textwidth]{gametree.pdf}
	\caption{The sequential truel game tree}
\end{figure}

Let us analyse all the different accuracy cases. We will only discuss the black
arrows of figure 1 since the green arrows indicate there is no choice. 

\subsection{If $p_a > p_b > p_c$} In the case where A kills B, A dies with
probability $p_c$.\\

If A misses B, B will have to choose between shooting A or C.
\begin{itemize}
	\item if B shoots A, B dies with probability $p_b*p_c$ because C will
		choose to shoot at the stronger A if he survives.
	\item if B shoots C, B will survive with probability 1 because C will
		choose to shoot A;
\end{itemize}
so B will shoot C, meaning that A will die with probability $(1-p_b)p_c$ if A
misses B.\\

So if A chooses to shoot B, A dies with probability 
$p_ap_c + (1-p_a)(1-p_b)p_c$\\

If A chooses to shoot C, A dies with probability $p_ap_b + (1-p_a)(1-p_b)p_c$
which is bigger because $p_b > p_c$; so the terminal histories part of the 
equilibria are \textbf{\{1, 2, 13, 9, 10\}}.

\subsection{If $p_a > p_c > p_b$} The reasoning is the same as above, but 
$p_c>p_b$ so the terminal histories part of the equilibrium are \textbf{\{14,
15\} and the histories corresponding to \{13, 9, 10\} in the subtree not shown}.

\subsection{If $p_b > p_a > p_c$} C will always choose to shoot B (in the
situations leading to terminal histories 7, 8, 11 and 12).\\

If A missed B and B chooses to shoot A, B will die with a probability of 
$p_bp_c + (1-p_b)p_c = p_c$. If B chooses to shoot C instead, B will die with
probability $(1-p_b)p_c$ which is smaller, so B will choose to shoot C.\\

So if A chooses to shoot B, A will die with probability $p_ap_c$ because if A
misses, B will shoot C, and if B misses, C will shoot B. However if A shoots C,
A dies with a probability $p_ap_b > p_ap_c$, so the terminal histories are 
\textbf{{\{1, 2, 13, 11, 12\}}}.

\subsection{If $p_b > p_c > p_a$} The reasoning is the exactly the same as 
above so the terminal histories are the same : \textbf{{\{1, 2, 13, 11, 12\}}}.

\subsection{If $p_c > p_b > p_a$} C will always choose to shoot B. Using the
reasoning above, it's easy to see that B will want to shoot C.\\

So A will die with probability $p_ap_c$ if he chooses to shoot B and $p_ap_b$
if he chooses to shoot C and $p_c>p_b$ so A will shoot C first and the terminal
histories are \textbf{\{14, 15\} and the histories corresponding to \{13, 11,
12\}}.

\subsection{If $p_c > p_a > p_b$} C will always shoot A. Using the reasoning 
from the first case, we know that B will shoot C because if he kills A, C
will shoot B and if he misses C, C will always shoot A.\\

So A will die with probability $p_ap_c + (1-p_b)p_c$ if he chooses to shoot B
and $p_ap_b + (1-p_b)p_c$ if he chooses to shoot C. So A will prefer to shoot C
because $p_c > p_b$. The terminal histories are \textbf{\{14, 15\} and the 
histories corresponding to \{13, 9, 10\} in the subtree not shown} (the same
as in the second case).

\subsection{Weakness is strength}
In all the cases where $p_c > p_b$ (C is stronger than B; e.g. cases 2, 5 and
6), A will choose to shoot C in the first place, and if B survives, B will 
decide to shoot at C every time. In fact, B is always better off shooting C,
but whenever C is stronger than B, since the outcome of missing B and missing C
is the same for A, the only difference is when A kills B or C. A will prefer
to shoot the player with the highest chance of hitting A so that A faces off 
the easiest opponent if A kills someone with his first bullet.

\subsection{Conclusion}
Since B will always shoot C if B survives, A will shoot at the most accurate
opponent to avoid facing him if he kills someone with his first bullet. It 
would have been faster to do the case analysis on each subgame rather than on
the whole game for each case.
\end{document}
